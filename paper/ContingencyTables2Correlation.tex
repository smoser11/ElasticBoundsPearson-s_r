% Options for packages loaded elsewhere
\PassOptionsToPackage{unicode}{hyperref}
\PassOptionsToPackage{hyphens}{url}
\PassOptionsToPackage{dvipsnames,svgnames,x11names}{xcolor}
%
\documentclass[
  12pt,
  letterpaper,
  DIV=11,
  numbers=noendperiod]{scrartcl}

\usepackage{amsmath,amssymb}
\usepackage{iftex}
\ifPDFTeX
  \usepackage[T1]{fontenc}
  \usepackage[utf8]{inputenc}
  \usepackage{textcomp} % provide euro and other symbols
\else % if luatex or xetex
  \usepackage{unicode-math}
  \defaultfontfeatures{Scale=MatchLowercase}
  \defaultfontfeatures[\rmfamily]{Ligatures=TeX,Scale=1}
\fi
\usepackage[]{palatino}
\ifPDFTeX\else  
    % xetex/luatex font selection
\fi
% Use upquote if available, for straight quotes in verbatim environments
\IfFileExists{upquote.sty}{\usepackage{upquote}}{}
\IfFileExists{microtype.sty}{% use microtype if available
  \usepackage[]{microtype}
  \UseMicrotypeSet[protrusion]{basicmath} % disable protrusion for tt fonts
}{}
\makeatletter
\@ifundefined{KOMAClassName}{% if non-KOMA class
  \IfFileExists{parskip.sty}{%
    \usepackage{parskip}
  }{% else
    \setlength{\parindent}{0pt}
    \setlength{\parskip}{6pt plus 2pt minus 1pt}}
}{% if KOMA class
  \KOMAoptions{parskip=half}}
\makeatother
\usepackage{xcolor}
\setlength{\emergencystretch}{3em} % prevent overfull lines
\setcounter{secnumdepth}{-\maxdimen} % remove section numbering
% Make \paragraph and \subparagraph free-standing
\ifx\paragraph\undefined\else
  \let\oldparagraph\paragraph
  \renewcommand{\paragraph}[1]{\oldparagraph{#1}\mbox{}}
\fi
\ifx\subparagraph\undefined\else
  \let\oldsubparagraph\subparagraph
  \renewcommand{\subparagraph}[1]{\oldsubparagraph{#1}\mbox{}}
\fi


\providecommand{\tightlist}{%
  \setlength{\itemsep}{0pt}\setlength{\parskip}{0pt}}\usepackage{longtable,booktabs,array}
\usepackage{calc} % for calculating minipage widths
% Correct order of tables after \paragraph or \subparagraph
\usepackage{etoolbox}
\makeatletter
\patchcmd\longtable{\par}{\if@noskipsec\mbox{}\fi\par}{}{}
\makeatother
% Allow footnotes in longtable head/foot
\IfFileExists{footnotehyper.sty}{\usepackage{footnotehyper}}{\usepackage{footnote}}
\makesavenoteenv{longtable}
\usepackage{graphicx}
\makeatletter
\def\maxwidth{\ifdim\Gin@nat@width>\linewidth\linewidth\else\Gin@nat@width\fi}
\def\maxheight{\ifdim\Gin@nat@height>\textheight\textheight\else\Gin@nat@height\fi}
\makeatother
% Scale images if necessary, so that they will not overflow the page
% margins by default, and it is still possible to overwrite the defaults
% using explicit options in \includegraphics[width, height, ...]{}
\setkeys{Gin}{width=\maxwidth,height=\maxheight,keepaspectratio}
% Set default figure placement to htbp
\makeatletter
\def\fps@figure{htbp}
\makeatother

\KOMAoption{captions}{tableheading}
\makeatletter
\@ifpackageloaded{caption}{}{\usepackage{caption}}
\AtBeginDocument{%
\ifdefined\contentsname
  \renewcommand*\contentsname{Table of contents}
\else
  \newcommand\contentsname{Table of contents}
\fi
\ifdefined\listfigurename
  \renewcommand*\listfigurename{List of Figures}
\else
  \newcommand\listfigurename{List of Figures}
\fi
\ifdefined\listtablename
  \renewcommand*\listtablename{List of Tables}
\else
  \newcommand\listtablename{List of Tables}
\fi
\ifdefined\figurename
  \renewcommand*\figurename{Figure}
\else
  \newcommand\figurename{Figure}
\fi
\ifdefined\tablename
  \renewcommand*\tablename{Table}
\else
  \newcommand\tablename{Table}
\fi
}
\@ifpackageloaded{float}{}{\usepackage{float}}
\floatstyle{ruled}
\@ifundefined{c@chapter}{\newfloat{codelisting}{h}{lop}}{\newfloat{codelisting}{h}{lop}[chapter]}
\floatname{codelisting}{Listing}
\newcommand*\listoflistings{\listof{codelisting}{List of Listings}}
\makeatother
\makeatletter
\makeatother
\makeatletter
\@ifpackageloaded{caption}{}{\usepackage{caption}}
\@ifpackageloaded{subcaption}{}{\usepackage{subcaption}}
\makeatother
\ifLuaTeX
  \usepackage{selnolig}  % disable illegal ligatures
\fi
\usepackage{bookmark}

\IfFileExists{xurl.sty}{\usepackage{xurl}}{} % add URL line breaks if available
\urlstyle{same} % disable monospaced font for URLs
\hypersetup{
  pdftitle={Marginal Extremes: Defining the Limits of Association in Cross-Tabulated Data},
  pdfauthor={Cees van der Eijk; Scott Moser},
  colorlinks=true,
  linkcolor={blue},
  filecolor={Maroon},
  citecolor={Blue},
  urlcolor={Blue},
  pdfcreator={LaTeX via pandoc}}

\title{Marginal Extremes: Defining the Limits of Association in
Cross-Tabulated Data}
\author{Cees van der Eijk \and Scott Moser}
\date{}

\begin{document}
\maketitle

\section{Optimal Coupling}\label{optimal-coupling}

\subsection{\texorpdfstring{\textbf{Proposition}}{Proposition}}\label{proposition}

Let \(X\) and \(Y\) be two ordinal random variables taking values in

\[\{1, 2, \ldots, K_X\} \quad \text{and} \quad \{1, 2, \ldots, K_Y\},\]

respectively. Suppose the \textbf{marginal distributions} of \(X\) and
\(Y\) are given (i.e., \(\Pr(X = i)\) is fixed for each
\(i\in \{1,\dots,K_X\}\), and \(\Pr(Y = j)\) is fixed for each
\(j\in \{1,\dots,K_Y\}\)). Among all joint distributions of \((X, Y)\)
consistent with these marginals, the \textbf{maximum value of Spearman's
rank-correlation} is attained via a comonotonic ordering -- by arranging
the highest ranks of \(X\) with the highest ranks of \(Y\). The
\textbf{minimum value of Spearman's rank-correlation} is attained via a
countermonotonic ordering -- by arranging the highest ranks of \(X\)
with the \emph{lowest} ranks of \(Y\).

Formally, if we define

\[\mathrm{rank}(X) = \begin{cases}  
1 & \text{if } X=1,\\  
2 & \text{if } X=2,\\  
\;\;\vdots\\  
K_X & \text{if } X=K_X,  
\end{cases}  
\quad  
\mathrm{rank}(Y) = \begin{cases}  
1 & \text{if } Y=1,\\  
2 & \text{if } Y=2,\\  
\;\;\vdots\\  
K_Y & \text{if } Y=K_Y,  
\end{cases}\]

then the joint distribution that \textbf{sorts} \(X\) in descending
order and \(Y\) in descending order (matching largest with largest,
next-largest with next-largest, etc.) maximizes

\[\rho_S(X, Y) \;=\; \mathrm{corr}\bigl(\mathrm{rank}(X), \mathrm{rank}(Y)\bigr).\]

\subsubsection{\texorpdfstring{\textbf{Proof}}{Proof}}\label{proof}

\begin{enumerate}
\def\labelenumi{\arabic{enumi}.}
\item
  \textbf{Spearman's Correlation as Pearson's Correlation of Ranks}\\
  By definition, Spearman's rank-correlation \(\rho_S(X,Y)\) is the
  Pearson correlation between \(\mathrm{rank}(X)\) and
  \(\mathrm{rank}(Y)\). That is,

  \[\rho_S(X, Y) \;=\; \frac{\mathrm{Cov}(\mathrm{rank}(X),\, \mathrm{rank}(Y))}{\sqrt{\mathrm{Var}(\mathrm{rank}(X))\;\mathrm{Var}(\mathrm{rank}(Y))}}.\]

  Maximizing \(\rho_S\) is equivalent to maximizing the \emph{expected
  product} \(\mathbb{E}[\mathrm{rank}(X)\,\mathrm{rank}(Y)]\) subject to
  the fixed marginal distributions of \(\mathrm{rank}(X)\) and
  \(\mathrm{rank}(Y)\).
\item
  \textbf{Rewriting the Expectation}\\
  Let \(r_1 < r_2 < \cdots < r_{K_X}\) be the distinct rank values for
  \(X\) and \(s_1 < s_2 < \cdots < s_{K_Y}\) be the distinct rank values
  for \(Y\). (Here \(r_i = i\) and \(s_j = j\) in typical usage.) Any
  joint distribution \(\Pr(X=i, Y=j)\) that respects \(\Pr(X=i)\) and
  \(\Pr(Y=j)\) must allocate probability mass in a 2D contingency table,
  but always summing to \(p_X(i)\) in row \(i\) and \(p_Y(j)\) in column
  \(j\). The quantity to be maximized is

  \[\sum_{i=1}^{K_X}\;\sum_{j=1}^{K_Y} \;r_i\,s_j \;\Pr(X = i, Y = j).\]
\item
  \textbf{Application of the Rearrangement Inequality}\\
  The \textbf{rearrangement inequality} (Hardy--Littlewood--Pólya,
  Theorem 368) or the discrete analog by Whitt (1978) tells us that for
  two finite sequences \(\{a_1,\dots,a_m\}\) and \(\{b_1,\dots,b_n\}\)
  (here, the sequences are effectively the rank values weighted by the
  probability masses), the sum of products \(\sum a_i\,b_{\sigma(i)}\)
  is maximized precisely when both sequences are sorted in the same
  order (both ascending or both descending). In the probability setting,
  ``sorting from largest to smallest'' means that the highest ranks of
  \(X\) should be paired with the highest ranks of \(Y\). Concretely, if
  \(X\) is in descending rank order \((K_X, K_X-1, \ldots)\) and \(Y\)
  is also in descending rank order \((K_Y, K_Y-1, \ldots)\), then the
  product of ranks \(\mathrm{rank}(X)\,\mathrm{rank}(Y)\) is as large as
  possible in expectation.
\item
  \textbf{Different Number of Levels}\\
  If \(K_X \neq K_Y\), the principle is the same. We let
  \(\mathrm{rank}(X)\in\{1,\dots,K_X\}\) and
  \(\mathrm{rank}(Y)\in\{1,\dots,K_Y\}\). The rearrangement inequality
  still applies: list all ``mass points'' of \(X\) in descending order
  of rank, and list all ``mass points'' of \(Y\) in descending order of
  rank. Pair them index-by-index so that the largest rank in \(X\) is
  matched with the largest rank in \(Y\). This arrangement yields the
  maximal expected product of ranks.
\item
  \textbf{Concrete Example (Different Levels)}

  \begin{itemize}
  \tightlist
  \item
    Suppose \(K_X = 3\) and \(K_Y = 4\). Then \(\mathrm{rank}(X)\) takes
    values \(\{1,2,3\}\) and \(\mathrm{rank}(Y)\) takes \(\{1,2,3,4\}\).
  \item
    Let \(\Pr(X=3) = 0.2\), \(\Pr(X=2)=0.5\), \(\Pr(X=1)=0.3\); and
    \(\Pr(Y=4)=0.1\), \(\Pr(Y=3)=0.4\), \(\Pr(Y=2)=0.3\),
    \(\Pr(Y=1)=0.2\).
  \item
    To maximize \(\mathbb{E}[\mathrm{rank}(X)\,\mathrm{rank}(Y)]\), we
    sort \(X\) from 3 down to 1 and \(Y\) from 4 down to 1. We then fill
    the contingency table so that the 0.2 probability mass of \(X=3\) is
    paired as much as possible with the 0.1 mass of \(Y=4\), the 0.4
    mass of \(Y=3\), and so on, always aligning the largest rank masses
    together.
  \item
    This yields the comonotonic distribution that achieves the largest
    \(\mathrm{rank}(X)\cdot \mathrm{rank}(Y)\) on average, and thus the
    maximum Spearman correlation.
  \end{itemize}
\item
  \textbf{Conclusion}\\
  By the rearrangement argument, the maximal Spearman rank-correlation
  is obtained via the comonotonic (descending-with-descending) joint
  distribution. Ties or repeated categories do not affect this
  principle, other than allowing multiple solutions that achieve the
  same maximum. Thus, the proposition is proved.
\end{enumerate}

\section{Extremal Values}\label{extremal-values}

We seek to find the maximum Pearson correlation coefficient,
\(r_{\max}\), between two discrete variables \(X\) and \(Y\) that take
values in \(\{0,1,2, \dots, K-1\}\), \textbf{given their fixed marginal
distributions}.

\begin{center}\rule{0.5\linewidth}{0.5pt}\end{center}

\subsubsection{\texorpdfstring{\textbf{Step 1: Define the Problem and
Notation}}{Step 1: Define the Problem and Notation}}\label{step-1-define-the-problem-and-notation}

We are given:

\begin{itemize}
\item
  \(n_X = (n_{X=0}, n_{X=1}, ..., n_{X=K-1})\), where \(n_{X=i}\) is the
  number of times \(X = i\) appears.
\item
  \(n_Y = (n_{Y=0}, n_{Y=1}, ..., n_{Y=K-1})\), where \(n_{Y=j}\) is the
  number of times \(Y = j\) appears.
\item
  The total number of observations:

  \[N = \sum_{i=0}^{K-1} n_{X=i} = \sum_{j=0}^{K-1} n_{Y=j}\]
\end{itemize}

\subsubsection{\texorpdfstring{\textbf{Step 2: Compute the Means and
Standard
Deviations}}{Step 2: Compute the Means and Standard Deviations}}\label{step-2-compute-the-means-and-standard-deviations}

The mean values of \(X\) and \(Y\) are:

\[\bar{X} = \frac{1}{N} \sum_{i=0}^{K-1} i \cdot n_{X=i}, \quad  
\bar{Y} = \frac{1}{N} \sum_{j=0}^{K-1} j \cdot n_{Y=j}\]

The variances are:

\[\sigma_X^2 = \frac{1}{N} \sum_{i=0}^{K-1} (i - \bar{X})^2 \cdot n_{X=i}, \quad  
\sigma_Y^2 = \frac{1}{N} \sum_{j=0}^{K-1} (j - \bar{Y})^2 \cdot n_{Y=j}\]

Thus, the standard deviations are:

\[\sigma_X = \sqrt{\frac{1}{N} \sum_{i=0}^{K-1} (i - \bar{X})^2 \cdot n_{X=i}}, \quad  
\sigma_Y = \sqrt{\frac{1}{N} \sum_{j=0}^{K-1} (j - \bar{Y})^2 \cdot n_{Y=j}}\]

\begin{center}\rule{0.5\linewidth}{0.5pt}\end{center}

\subsubsection{\texorpdfstring{\textbf{Step 3: Construct the Joint
Distribution for Maximum
Correlation}}{Step 3: Construct the Joint Distribution for Maximum Correlation}}\label{step-3-construct-the-joint-distribution-for-maximum-correlation}

To maximize \(r\), we must maximize:

\[\text{Cov}(X,Y) = E[XY] - \bar{X} \bar{Y}\]

To do this, we construct a \textbf{Sorted Array}:

\begin{enumerate}
\def\labelenumi{\arabic{enumi}.}
\tightlist
\item
  Sort the values of \(X\) in descending order according to their
  frequencies.
\item
  Independently, sort the values of \(Y\) in descending order according
  to their frequencies.
\item
  Assign pairings \((X_i, Y_i)\) in order, ensuring that the highest
  values of \(X\) are paired with the highest values of \(Y\), while
  respecting the marginal totals.
\end{enumerate}

Let \(m_{i,j}\) denote the number of times the pair \((i,j)\) appears.
The optimal strategy follows:

\[m_{i,j} = \min(n_{X=i}, n_{Y=j})\]

This ensures that the highest available values of \(X\) and \(Y\) are
paired together as much as possible.

\begin{center}\rule{0.5\linewidth}{0.5pt}\end{center}

\subsubsection{\texorpdfstring{\textbf{Step 4: Compute
\(E[XY]\)}}{Step 4: Compute E{[}XY{]}}}\label{step-4-compute-exy}

Given the optimal pair assignments:

\[E[XY]_{\max} = \frac{1}{N} \sum_{i=0}^{K-1} \sum_{j=0}^{K-1} i \cdot j \cdot m_{i,j}\]

Substituting \(m_{i,j} = \min(n_{X=i}, n_{Y=j})\), we obtain:

\[E[XY]_{\max} = \frac{1}{N} \sum_{i=0}^{K-1} i \sum_{j=0}^{K-1} j \cdot \min(n_{X=i}, n_{Y=j})\]

\begin{center}\rule{0.5\linewidth}{0.5pt}\end{center}

\subsubsection{\texorpdfstring{\textbf{Step 5: Compute Maximum
Covariance}}{Step 5: Compute Maximum Covariance}}\label{step-5-compute-maximum-covariance}

\[\text{Cov}_{\max}(X,Y) = E[XY]_{\max} - \bar{X} \bar{Y}\]

Substituting the expectation:

\[\text{Cov}_{\max}(X,Y) = \frac{1}{N} \sum_{i=0}^{K-1} i \sum_{j=0}^{K-1} j \cdot \min(n_{X=i}, n_{Y=j}) - \bar{X} \bar{Y}\]

\begin{center}\rule{0.5\linewidth}{0.5pt}\end{center}

\subsubsection{\texorpdfstring{\textbf{Step 6: Compute
\(r_{\max}\)}}{Step 6: Compute r\_\{\textbackslash max\}}}\label{step-6-compute-r_max}

Using Pearson's formula:

\[r_{\max} = \frac{\text{Cov}_{\max}(X, Y)}{\sigma_X \sigma_Y}\]
\[r_{\max} =  
\frac{\frac{1}{N} \sum_{i=0}^{K-1} i \sum_{j=0}^{K-1} j \cdot \min(n_{X=i}, n_{Y=j}) - \bar{X} \bar{Y}}  
{\sigma_X \sigma_Y}\]

\begin{center}\rule{0.5\linewidth}{0.5pt}\end{center}

\begin{center}\rule{0.5\linewidth}{0.5pt}\end{center}

\subsection{\texorpdfstring{\textbf{Final
Answer}}{Final Answer}}\label{final-answer}

\[r_{\max} =  
\frac{\frac{1}{N} \sum_{i=0}^{K-1} i \sum_{j=0}^{K-1} j \cdot \min(n_{X=i}, n_{Y=j}) - \bar{X} \bar{Y}}  
{\sigma_X \sigma_Y}\]

This refined response now properly defines the indices in summations,
explains the sorting approach, and suggests adding a numerical example
for clarity.

\begin{proposition}
Let $X$ and $Y$


\end{proposition}

\subsection{Proposition: Maximum Spearman Rank
Correlation}\label{proposition-maximum-spearman-rank-correlation}

Let \(X\) and \(Y\) be two discrete ordinal random variables, taking
values in:

\begin{itemize}
\tightlist
\item
  \(X \in \{x_1, x_2, \dots, x_m\}\) with known marginal frequencies
  \((n_{X=x_1}, n_{X=x_2}, ..., n_{X=x_m})\), where \(n_{X=x_i}\)
  denotes the number of observations where \(X = x_i\).
\item
  \(Y \in \{y_1, y_2, \dots, y_n\}\) with known marginal frequencies
  \((n_{Y=y_1}, n_{Y=y_2}, ..., n_{Y=y_n})\).
\end{itemize}

The \textbf{maximum Spearman rank correlation} is obtained by
constructing the joint distribution as follows:

\begin{enumerate}
\def\labelenumi{\arabic{enumi}.}
\tightlist
\item
  \textbf{Sort} the values of \(X\) in descending order according to
  their frequencies.
\item
  \textbf{Independently sort} the values of \(Y\) in descending order
  according to their frequencies.
\item
  \textbf{Pair} the highest values of \(X\) with the highest values of
  \(Y\), the second highest with the second highest, and so on, while
  respecting the marginal totals.
\end{enumerate}

That is, the rank vectors \(R_X\) and \(R_Y\) should be
\textbf{similarly ordered} to maximize Spearman rank correlation.

\begin{center}\rule{0.5\linewidth}{0.5pt}\end{center}

\subsection{\texorpdfstring{\textbf{Step 1: Definition of Spearman Rank
Correlation}}{Step 1: Definition of Spearman Rank Correlation}}\label{step-1-definition-of-spearman-rank-correlation}

Spearman's rank correlation coefficient is given by:

\[\rho = \frac{\sum_{i=1}^{N} R_{X,i} R_{Y,i} - N \bar{R}_X \bar{R}_Y}{N \sigma_{R_X} \sigma_{R_Y}}\]

where:

\begin{itemize}
\tightlist
\item
  \(N\) is the total number of observations,
\item
  \(R_{X,i}\) and \(R_{Y,i}\) are the ranks of the \(i\)-th observations
  of \(X\) and \(Y\),
\item
  \(\bar{R}_X\) and \(\bar{R}_Y\) are the mean ranks,
\item
  \(\sigma_{R_X}\) and \(\sigma_{R_Y}\) are the standard deviations of
  the rank variables.
\end{itemize}

To maximize \(\rho\), we must maximize:

\[\sum_{i=1}^{N} R_{X,i} R_{Y,i}\]

which is equivalent to maximizing the \textbf{sum of products of ranks}.

\begin{center}\rule{0.5\linewidth}{0.5pt}\end{center}

\subsection{\texorpdfstring{\textbf{Step 2: Application of the
Hardy-Littlewood-Polya Rearrangement
Inequality}}{Step 2: Application of the Hardy-Littlewood-Polya Rearrangement Inequality}}\label{step-2-application-of-the-hardy-littlewood-polya-rearrangement-inequality}

The Hardy-Littlewood-Polya rearrangement inequality (Theorem 368 in
\emph{Inequalities}) states:

\[\sum_{i=1}^{N} a_i^* b_i^* \geq \sum_{i=1}^{N} a_i b_i \geq \sum_{i=1}^{N} a_i^* b_i'\]

where:

\begin{itemize}
\tightlist
\item
  \(a_i^*\) and \(b_i^*\) are the \textbf{similarly ordered}
  permutations of \(a_i\) and \(b_i\) (largest with largest, second
  largest with second largest, etc.).
\item
  \(a_i^*\) and \(b_i'\) are \textbf{oppositely ordered} permutations
  (largest with smallest, second largest with second smallest, etc.).
\end{itemize}

Applying this inequality to the rank variables \(R_X\) and \(R_Y\), we
conclude that:

\[\sum_{i=1}^{N} R_{X,i} R_{Y,i}\]

is maximized when \(R_X\) and \(R_Y\) are \textbf{sorted in the same
order}. This ensures that the sum of rank products is maximized, and
thus Spearman's \(\rho\) is maximized.

\begin{center}\rule{0.5\linewidth}{0.5pt}\end{center}

\subsection{\texorpdfstring{\textbf{Step 3: Constructing the Optimal
Joint
Distribution}}{Step 3: Constructing the Optimal Joint Distribution}}\label{step-3-constructing-the-optimal-joint-distribution}

To enforce this optimal alignment, we follow these steps:

\begin{enumerate}
\def\labelenumi{\arabic{enumi}.}
\tightlist
\item
  \textbf{Sort} \(X\) from largest to smallest according to its marginal
  frequencies.
\item
  \textbf{Independently sort} \(Y\) from largest to smallest according
  to its marginal frequencies.
\item
  \textbf{Pair} values such that the highest available ranks of \(X\)
  are assigned to the highest available ranks of \(Y\), and continue in
  this fashion until all observations are assigned.
\end{enumerate}

This ensures that the largest ranks of \(X\) are aligned with the
largest ranks of \(Y\), which is the necessary condition for maximizing
Spearman's correlation.

\begin{center}\rule{0.5\linewidth}{0.5pt}\end{center}

\subsection{\texorpdfstring{\textbf{Step 4: Numerical
Example}}{Step 4: Numerical Example}}\label{step-4-numerical-example}

Let's consider a small example where \(X\) and \(Y\) have different
numbers of levels.

\textbf{Marginals:}

\begin{itemize}
\tightlist
\item
  \(X\) takes values \(\{0,1,2\}\) with frequencies \((5,3,2)\).
\item
  \(Y\) takes values \(\{0,1,2,3\}\) with frequencies \((3,4,2,1)\).
\end{itemize}

\subsubsection{\texorpdfstring{\textbf{Step 1: Sort by
Frequency}}{Step 1: Sort by Frequency}}\label{step-1-sort-by-frequency}

Sorting both in decreasing order:

\[X: (0,0,0,0,0,1,1,1,2,2)\] \[Y: (1,1,1,1,0,0,2,2,3,3)\]

\subsubsection{\texorpdfstring{\textbf{Step 2:
Pairing}}{Step 2: Pairing}}\label{step-2-pairing}

Pairing elements sequentially, ensuring highest values are together:

\[(X,Y) = [(0,1), (0,1), (0,1), (0,1), (0,0), (1,0), (1,2), (1,2), (2,3), (2,3)]\]

\subsubsection{\texorpdfstring{\textbf{Step 3: Compute Spearman's Rank
Correlation}}{Step 3: Compute Spearman's Rank Correlation}}\label{step-3-compute-spearmans-rank-correlation}

Assigning ranks and computing \(\rho\) shows that this maximized the sum
\(\sum R_{X,i} R_{Y,i}\), thus achieving \(\rho_{\max}\).

\begin{center}\rule{0.5\linewidth}{0.5pt}\end{center}

\subsection{\texorpdfstring{\textbf{Conclusion}}{Conclusion}}\label{conclusion}

By applying the Hardy-Littlewood-Polya inequality, we rigorously prove
that the maximum Spearman correlation is achieved when the joint
distribution is constructed by sorting \(X\) and \(Y\) separately in
decreasing order and then pairing them. This construction ensures that
rank orderings are maximally similar, which, by definition, maximizes
the Spearman rank correlation.

\[\square\]

\section{Correlary: The Min and Max
Values}\label{correlary-the-min-and-max-values}

We seek to find the maximum Pearson correlation coefficient,
\(r_{\max}\), between two discrete variables \(X\) and \(Y\) that take
values in \(\{0,1,2, \dots, K-1\}\), \textbf{given their fixed marginal
distributions}.

\begin{center}\rule{0.5\linewidth}{0.5pt}\end{center}

\subsubsection{\texorpdfstring{\textbf{Step 1: Define the Problem and
Notation}}{Step 1: Define the Problem and Notation}}\label{step-1-define-the-problem-and-notation-1}

We are given:

\begin{itemize}
\item
  \(n_X = (n_{X=0}, n_{X=1}, ..., n_{X=K-1})\), where \(n_{X=i}\) is the
  number of times \(X = i\) appears.
\item
  \(n_Y = (n_{Y=0}, n_{Y=1}, ..., n_{Y=K-1})\), where \(n_{Y=j}\) is the
  number of times \(Y = j\) appears.
\item
  The total number of observations:

  \[N = \sum_{i=0}^{K-1} n_{X=i} = \sum_{j=0}^{K-1} n_{Y=j}\]
\end{itemize}

\subsubsection{\texorpdfstring{\textbf{Step 2: Compute the Means and
Standard
Deviations}}{Step 2: Compute the Means and Standard Deviations}}\label{step-2-compute-the-means-and-standard-deviations-1}

The mean values of \(X\) and \(Y\) are:

\[\bar{X} = \frac{1}{N} \sum_{i=0}^{K-1} i \cdot n_{X=i}, \quad  
\bar{Y} = \frac{1}{N} \sum_{j=0}^{K-1} j \cdot n_{Y=j}\]

The variances are:

\[\sigma_X^2 = \frac{1}{N} \sum_{i=0}^{K-1} (i - \bar{X})^2 \cdot n_{X=i}, \quad  
\sigma_Y^2 = \frac{1}{N} \sum_{j=0}^{K-1} (j - \bar{Y})^2 \cdot n_{Y=j}\]

Thus, the standard deviations are:

\[\sigma_X = \sqrt{\frac{1}{N} \sum_{i=0}^{K-1} (i - \bar{X})^2 \cdot n_{X=i}}, \quad  
\sigma_Y = \sqrt{\frac{1}{N} \sum_{j=0}^{K-1} (j - \bar{Y})^2 \cdot n_{Y=j}}\]

\begin{center}\rule{0.5\linewidth}{0.5pt}\end{center}

\subsubsection{\texorpdfstring{\textbf{Step 3: Construct the Joint
Distribution for Maximum
Correlation}}{Step 3: Construct the Joint Distribution for Maximum Correlation}}\label{step-3-construct-the-joint-distribution-for-maximum-correlation-1}

To maximize \(r\), we must maximize:

\[\text{Cov}(X,Y) = E[XY] - \bar{X} \bar{Y}\]

To do this, we construct a \textbf{Sorted Array}:

\begin{enumerate}
\def\labelenumi{\arabic{enumi}.}
\tightlist
\item
  Sort the values of \(X\) in descending order according to their
  frequencies.
\item
  Independently, sort the values of \(Y\) in descending order according
  to their frequencies.
\item
  Assign pairings \((X_i, Y_i)\) in order, ensuring that the highest
  values of \(X\) are paired with the highest values of \(Y\), while
  respecting the marginal totals.
\end{enumerate}

Let \(m_{i,j}\) denote the number of times the pair \((i,j)\) appears.
The optimal strategy follows:

\[m_{i,j} = \min(n_{X=i}, n_{Y=j})\]

This ensures that the highest available values of \(X\) and \(Y\) are
paired together as much as possible.

\begin{center}\rule{0.5\linewidth}{0.5pt}\end{center}

\subsubsection{\texorpdfstring{\textbf{Step 4: Compute
\(E[XY]\)}}{Step 4: Compute E{[}XY{]}}}\label{step-4-compute-exy-1}

Given the optimal pair assignments:

\[E[XY]_{\max} = \frac{1}{N} \sum_{i=0}^{K-1} \sum_{j=0}^{K-1} i \cdot j \cdot m_{i,j}\]

Substituting \(m_{i,j} = \min(n_{X=i}, n_{Y=j})\), we obtain:

\[E[XY]_{\max} = \frac{1}{N} \sum_{i=0}^{K-1} i \sum_{j=0}^{K-1} j \cdot \min(n_{X=i}, n_{Y=j})\]

\begin{center}\rule{0.5\linewidth}{0.5pt}\end{center}

\subsubsection{\texorpdfstring{\textbf{Step 5: Compute Maximum
Covariance}}{Step 5: Compute Maximum Covariance}}\label{step-5-compute-maximum-covariance-1}

\[\text{Cov}_{\max}(X,Y) = E[XY]_{\max} - \bar{X} \bar{Y}\]

Substituting the expectation:

\[\text{Cov}_{\max}(X,Y) = \frac{1}{N} \sum_{i=0}^{K-1} i \sum_{j=0}^{K-1} j \cdot \min(n_{X=i}, n_{Y=j}) - \bar{X} \bar{Y}\]

\begin{center}\rule{0.5\linewidth}{0.5pt}\end{center}

\subsubsection{\texorpdfstring{\textbf{Step 6: Compute
\(r_{\max}\)}}{Step 6: Compute r\_\{\textbackslash max\}}}\label{step-6-compute-r_max-1}

Using Pearson's formula:

\[r_{\max} = \frac{\text{Cov}_{\max}(X, Y)}{\sigma_X \sigma_Y}\]
\[r_{\max} =  
\frac{\frac{1}{N} \sum_{i=0}^{K-1} i \sum_{j=0}^{K-1} j \cdot \min(n_{X=i}, n_{Y=j}) - \bar{X} \bar{Y}}  
{\sigma_X \sigma_Y}\]

\begin{center}\rule{0.5\linewidth}{0.5pt}\end{center}

\subsubsection{\texorpdfstring{\textbf{Step 7: Numerical Example (for
Clarity)}}{Step 7: Numerical Example (for Clarity)}}\label{step-7-numerical-example-for-clarity}

Suppose \(K=3\), and the marginals are:

\[n_X = (10, 5, 3), \quad n_Y = (8, 7, 3)\]

So:

\begin{itemize}
\tightlist
\item
  \(N = 18\)
\item
  Compute \(\bar{X}\), \(\bar{Y}\), \(E[XY]_{\max}\), and then
  \(r_{\max}\) explicitly.
\end{itemize}

By providing such an example, the reader can verify the calculations
with concrete numbers.

\begin{center}\rule{0.5\linewidth}{0.5pt}\end{center}

\subsection{\texorpdfstring{\textbf{Final
Answer}}{Final Answer}}\label{final-answer-1}

\[r_{\max} =  
\frac{\frac{1}{N} \sum_{i=0}^{K-1} i \sum_{j=0}^{K-1} j \cdot \min(n_{X=i}, n_{Y=j}) - \bar{X} \bar{Y}}  
{\sigma_X \sigma_Y}\]

\section{Fréchet--Hoeffding Bounds}\label{fruxe9chethoeffding-bounds}

\subsection{Symmetry}\label{symmetry}



\end{document}
