\documentclass[12pt]{article}
\usepackage{amsmath, amssymb}
\usepackage{graphicx}
\usepackage{booktabs}
\usepackage{geometry}
\usepackage{setspace}
\usepackage{hyperref}
\usepackage{enumitem}
\usepackage[backend=biber, style=apa]{biblatex}
\addbibresource{references.bib}

\geometry{margin=1in}
\setstretch{1.25}

\title{Bounding Pearson’s Correlation for Ordinal Data: Theory, Symmetry, and Empirical Consequences}
\author{Author names omitted for review}
\date{}

\begin{document}

\maketitle

\begin{abstract}
Pearson’s correlation coefficient \( r \) is commonly used with ordinal variables, despite being designed for interval-scale data. This paper develops a theoretical framework for bounding \( r \) given fixed marginal distributions. Using rearrangement inequalities and connections to Fréchet bounds, we derive conditions for achieving the minimum and maximum correlations, show when symmetry fails (\( r_{\min} \ne -r_{\max} \)), and prove that the expected value of \( r \) under randomization is zero. Simulations illustrate how marginal skew and asymmetry influence the attainable range of \( r \), with implications for factor analysis and structural modeling.
\end{abstract}

\tableofcontents
\newpage

%-------------------------------------------
\section{Introduction}

\begin{itemize}
  \item Pearson’s \( r \) is widely used even when the variables are ordinal.
  \item Ordinal variables violate the assumptions of interval measurement, yet researchers apply \( r \) due to its familiarity and convenience.
  \item But \( r \) is constrained by the marginals of the two variables — its range is not always \([-1, 1]\).
  \item This paper develops the theoretical and empirical framework to:
    \begin{enumerate}[label=(\roman*)]
      \item Derive the min and max of \( r \) for fixed marginals
      \item Relate these to classical probabilistic bounds (Fréchet–Hoeffding)
      \item Identify when bounds are symmetric and when they are not
      \item Prove that \( \mathbb{E}[r] = 0 \) under randomization
      \item Illustrate implications for interpreting correlation-based analyses
    \end{enumerate}
\end{itemize}

%-------------------------------------------
\section{Setup and Notation}

\begin{itemize}
  \item Define ordinal variables \( X \) and \( Y \) with \( K_X, K_Y \) categories.
  \item Let \( p_i = P(X = x_i), q_j = P(Y = y_j) \); fixed marginals.
  \item Pearson’s correlation:
  \[
  r = \frac{\mathbb{E}[XY] - \mathbb{E}[X]\mathbb{E}[Y]}{\sqrt{\text{Var}(X)\text{Var}(Y)}}
  \]
  \item Our goal: Determine the extreme values of \( r \) over all joint distributions consistent with \( \{p_i\}, \{q_j\} \).
\end{itemize}

%-------------------------------------------
\section{Theory: Bounding Pearson’s \( r \) with Fixed Marginals}

\subsection{(1) Finding the Extremal Joint Distributions}
\begin{itemize}
  \item Define the optimization problem: maximize/minimize \( \mathbb{E}[XY] \) over all couplings.
  \item Use rearrangement inequalities and greedy coupling to construct \( F_{\max}(x, y) \).
  \item Show examples with 4×5 and 5×5 marginals.
\end{itemize}

\subsection{(1a) Closed-Form Expressions}
\begin{itemize}
  \item Present analytic expressions for the min and max of \( \mathbb{E}[XY] \), where derivable.
  \item Discuss bounds when exact forms are not attainable due to discreteness.
\end{itemize}

\subsection{(2) Relationship to Fréchet–Hoeffding Bounds}
\begin{itemize}
  \item Show that min/max \( r \) corresponds to extremal cases of joint distribution under Fréchet bounds.
  \item Link this to the feasible region defined by Boole–Fréchet inequalities.
\end{itemize}

\subsection{(3) Symmetry, Asymmetry, and the Role of Ties}
\begin{itemize}
  \item Theoretical result: symmetry holds under symmetric marginals.
  \item When marginals are asymmetric or tied, \( r_{\min} \ne -r_{\max} \).
  \item Explore how entropy of marginals predicts symmetry breaking.
\end{itemize}

\subsection{(4) When Can \( r_{\min} > 0 \)?}
\begin{itemize}
  \item Conditions under which the lower bound on \( r \) is strictly positive.
  \item Example simulations to show how skewed marginals can eliminate negative correlation.
\end{itemize}

\subsection{(5) Center of the Randomization Distribution}
\begin{itemize}
  \item Proof: under uniform random permutation (null of independence), \( \mathbb{E}[r] = 0 \), regardless of marginals.
  \item Use this to justify randomization tests even when marginals are highly asymmetric.
\end{itemize}

%-------------------------------------------
\section{Empirical Illustrations: Simulating the Randomization Distribution}

\subsection{Simulation Design}
\begin{itemize}
  \item Fixed-marginal randomization using permutation methods (existing R code).
  \item Tables: 4×4, 4×5, 3×7, 5×5, 7×10.
  \item Varying marginal shape: symmetric, skewed, uniform, bimodal.
\end{itemize}

\subsection{Observed Range of \( r \) under the Null}
\begin{itemize}
  \item Show full distribution of \( r \) under permutation.
  \item Overlay theoretical min/max — often quite narrow.
  \item Demonstrate how symmetry or skew influences spread and center.
\end{itemize}

\subsection{When Is \( r \) Meaningful?}
\begin{itemize}
  \item Compare observed correlations from real data to attainable range under null.
  \item Situations where a modest \( r \) (e.g., .25) is near-maximal vs. those where it’s central.
  \item Implications for interpreting correlation-based results in practice.
\end{itemize}

%-------------------------------------------
\section{Implications for Applied Work}

\subsection{Consequences for Statistical Modeling}
\begin{itemize}
  \item SEM, factor analysis: attenuated \( r \) leads to over-factoring or underestimation.
  \item Justification for polychoric correlation vs. bounded Pearson’s \( r \).
\end{itemize}

\subsection{Recommendations}
\begin{itemize}
  \item Check whether observed \( r \) is near the bounds given marginals.
  \item Report bounds alongside point estimates.
  \item Use permutation testing or bounds-aware diagnostics for correlation interpretation.
\end{itemize}

%-------------------------------------------
\section{Conclusion}
\begin{itemize}
  \item Pearson’s correlation has hidden constraints when applied to ordinal data.
  \item We derive sharp theoretical bounds and show when and why symmetry breaks.
  \item Our results support more nuanced interpretation of correlation in ordinal contexts.
  \item A lightweight R package accompanies this paper, allowing users to compute bounds and simulate randomization distributions.
\end{itemize}

%-------------------------------------------
\appendix
\section*{Technical Appendix (Optional)}
\begin{itemize}
  \item Formal derivations of min/max \( E[XY] \).
  \item Symmetry-breaking conditions.
  \item Greedy algorithm for coupling construction (pseudocode).
\end{itemize}

\printbibliography

\end{document}
